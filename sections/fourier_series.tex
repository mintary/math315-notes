\chapter{Fourier series}

Fourier series are a way to represent a periodic function as an infinite sum of sine and cosine functions. A function $f$ is said to be \textbf{periodic} with period $2L$ if it satisfies the following condition:
\begin{equation*}
f(t+2L) = f(t), \quad \forall t \in \mathbb{R}
\end{equation*}

The timepoints at which a value reoccurs $4L, 8L \ldots$ are also called \textbf{harmonics}.

Any window of width $2L$ will define the function for all $t$. We build up to this interval from the cosine and sine functions, which are defined as their behaviour over the window $\left[ - \pi , \pi \right]$.

\nt{
    $\cos (m t)$ has a period of $\frac{2 \pi}{m}$, and $\sin (m t)$ has a period of $\frac{2 \pi}{m}$. Furthermore, if $f(t)$ and $g(t)$ are periodic functions with period $2L$, then the function $h(t) = a f(t) + b g(t)$ is also periodic with period $2L$ for any constants $a$ and $b$. This is because:
\begin{align*}
h(t + 2L) &= a f(t + 2L) + b g(t + 2L) \\
&= a f(t) + b g(t) \\
&= h(t)
\end{align*}
}

\section{Fourier coefficients}
A Fourier series has the form:
\begin{equation*}
\boxed{
    f(t) = \frac{a_0}{2} + \sum_{m=1}^{\infty} \left( a_m \cos \left( mt \right) + b_m \sin \left( mt \right) \right)
}
\end{equation*}

The coefficients $a_0, a_m, b_m$ are called the \textbf{Fourier coefficients} of the function $f$.

Intuitively, for any given point $f(t*)$, we can think of the Fourier series as a sum of infinitely many sine and cosine waves that add up to $f(t*)$. The Fourier coefficients determine the amplitude of each sine and cosine wave in the series. The more terms we include in the series, the closer the approximation will be to the original function $f(t)$.

We wish to define the Fourier coefficients so that we can reconstruct the original function.

We begin by considering $a_0$. We define the function:
\begin{equation*}
\text{Average}(f) = \frac{1}{2\pi} \int_{-\pi}^{\pi} f(t) dt
\end{equation*}

Effectively, this is the average value of the function $f$ over one period. We can also write this as:
\begin{equation*}
\text{Average}(f) = \frac{1}{2L} \int_{-L}^{L} f(t) dt
\end{equation*}

The average value of $\cos(mt)$ and $\sin(mt)$ over one period is zero, so we have:
\begin{equation*}
\text{Average}(f) = \frac{a_0}{2}
\end{equation*}

Thus, we can define $a_0$ as:
\begin{equation*}
\boxed{
a_0 = \frac{1}{2 \pi} \int_{-\pi}^{\pi} f(t) dt
}
\end{equation*}

\subsection{Properties of \texorpdfstring{$\sin$ and $\cos$}{sin and cos} integrals}

The following can be obtained by using trig identities, converting to complex exponentials, or integration by parts.

\begin{equation*}
\int_{-\pi}^{\pi} \cos(mt) \sin(mt) dt = 0
\end{equation*}

\begin{equation*}
\int_{-\pi}^{\pi} \cos(mt) \cos(nt) dt = \begin{cases}
0 & m \neq n \\
\pi & m = n \neq 0 \\
2\pi & m = n = 0
\end{cases}
\end{equation*}

\begin{equation*}
\int_{-\pi}^{\pi} \sin(mt) \sin(nt) dt = \begin{cases}
0 & m \neq n \\
\pi & m = n \neq 0 \\
0 & m = n = 0
\end{cases}
\end{equation*}

In general, we can compute the integrals by also considering odd and even functions. For example, $\cos(mt) \sin(nt)$ is an odd function, so its integral over $[-\pi, \pi]$ is zero.

\subsection{Deriving the Fourier coefficients}
Intuitively, $a_m$ and $b_m$ tell us "how much" of that term is present in the function. 

To derive $a_m$, we use the "average" function again, and integrate over the period:
\begin{align*}
a_m = \frac{1}{\pi} \int_{-\pi}^{\pi} f(t) \cos(mt) dt
\end{align*}

The multiplication by $\cos(mt)$ allows us to "pick out" the $a_m$ term from the Fourier series, since the integrals of $\cos(mt) \cos(nt)$ and $\sin(mt) \cos(nt)$ are zero for $m \neq n$. This reflects the orthogonality of the sine and cosine functions. 

\begin{equation*}
\int_{-\pi}^{\pi} a_m \cos(mt) \cos(nt) dt = \begin{cases}
0 & m \neq n \\
\pi a_m & m = n \neq 0 \\
2\pi a_0 & m = n = 0
\end{cases}
\end{equation*}


Similarly, we can derive $b_m$ as:
\begin{align*}
b_m = \frac{1}{\pi} \int_{-\pi}^{\pi} f(t) \sin(mt) dt
\end{align*}

\begin{equation*}
\int_{\pi}^{\pi} b_m \sin(mt) \sin(nt) dt = \begin{cases}
0 & m \neq n \\
\pi b_m & m = n \neq 0 \\
0 & m = n = 0
\end{cases}
\end{equation*}

Then the \textbf{only} nonzero terms occur when $m = n \neq 0$.

\begin{equation*}
\boxed{
    a_m = \frac{1}{\pi} \int_{-\pi}^{\pi} f(t) \cos(mt) dt, \quad b_m = \frac{1}{\pi} \int_{-\pi}^{\pi} f(t) \sin(mt) dt
}
\end{equation*}

By change of variables (on $t$), we can also write the Fourier coefficients as:
\begin{equation*}
\boxed{
    a_m = \frac{1}{L} \int_{-L}^{L} f(t) \cos\left( \frac{m \pi t}{L} \right) dt, \quad b_m = \frac{1}{L} \int_{-L}^{L} f(t) \sin\left( \frac{m \pi t}{L} \right) dt
}
\end{equation*}

This allows us to have the function defined on any interval of length $2L$, not just $[-\pi, \pi]$.

\subsection{Functions defined as Fourier series}

Let us consider the \textbf{square wave} function defined as:
\begin{equation*}
f(t) = \begin{cases}
1 & 0 < t < \pi \\
-1 & -\pi < t < 0
\end{cases}
\end{equation*}

Notice that this function is odd.

We can compute the Fourier coefficients, starting with the constant $a_0$ term:
\begin{equation*}
a_0 = \frac{1}{\pi} \int_{-\pi}^{\pi} f(t) dt = 0
\end{equation*}

Then, since $f$ is odd, we have $a_m = 0$ for all $m$. We can compute $b_m$ as follows:
\begin{align*}
b_m &= \frac{1}{\pi} \int_{-\pi}^{\pi} f(t) \sin(mt) dt \\
&= 2 \int_{0}^{\pi} f(t) \sin(mt) dt \\
&= \frac{2}{\pi} \int_{0}^{\pi} \sin(mt) dt \\
&= \frac{2}{\pi} \left[ -\frac{\cos(mt)}{m} \right]_{0}^{\pi} \\
&= \frac{2}{m \pi} \left[ -\cos(m \pi) + 1 \right] \\
&= \frac{2}{m \pi} \left[ 1 - \cos(m \pi) \right] 
\end{align*}

As a table:
\begin{center}
\begin{tabular}{c|c|c}
$m$ & $\cos(m \pi)$ & $1 - \cos(m \pi)$ \\
\hline
1 & -1 & 2 \\
2 & 1 & 0 \\
3 & -1 & 2 \\
4 & 1 & 0 \\
5 & -1 & 2 \\
\vdots & \vdots & \vdots
\end{tabular}
\end{center}

Then we have that:
\begin{equation*}
b_m = \begin{cases}
\frac{4}{m \pi} & m \text{ odd} \\
0 & m \text{ even}
\end{cases}
\end{equation*}

Thus, the Fourier series for the square wave function is:
\begin{equation*}
f(t) = \frac{4}{\pi} \left( \sin(t) + \frac{1}{3} \sin(3t) + \frac{1}{5} \sin(5t) + \cdots \right)
\end{equation*}

\subsubsection{Continuity, differentiability, and rate of decay of terms}

Consider different waves: a square wave, triangle wave, and a wave consisting of parabolas.
\begin{center}
\includegraphics[width=0.4\textwidth]{images/fourier_series/waveformshapes.png}
\end{center}

The square wave is discontinuous, the triangle wave is continuous but not differentiable, and the parabola wave is continuous and differentiable.

The Fourier coefficients of the square wave decay as $\frac{1}{m}$, the Fourier coefficients of the triangle wave decay as $\frac{1}{m^2}$, and the Fourier coefficients of the parabola wave decay as $\frac{1}{m^3}$. This occurs because the Fourier coefficients are related to the smoothness of the function. The smoother the function, the faster the Fourier coefficients decay. In general, if a function is $k$ times differentiable, then its Fourier coefficients decay at least as fast as $\frac{1}{m^{k+1}}$.

\nt{
    Piecewise continuous functions are functions that are continuous on each piece of their domain, but may have a finite number of jump discontinuities. For example, the square wave function is piecewise continuous, as it is continuous on the intervals $(-\pi, 0)$ and $(0, \pi)$, but has a jump discontinuity at $t = 0$.

    We can define the left and right limits of a function $f$ at a point $a$ as follows:
    \begin{equation*}
    f(a^{-}) = \lim_{t \to a^{-}} f(t), \quad f(a^{+}) = \lim_{t \to a^{+}} f(t)
    \end{equation*}
    A \textbf{nice function} is defined if we have $f(a) = \frac{1}{2} \left( f(a^{-}) + f(a^{+}) \right)$ for all $a$. In other words, the function is defined at the point of discontinuity as the average of the left and right limits. If a function is nice, then its Fourier series converges to the function at all points.
}

\subsection{Manipulating Fourier series}

\begin{equation*}
\boxed{
    \sin(\theta + \frac{\pi}{2}) = \cos(\theta), \quad \cos(\theta + \frac{\pi}{2}) = -\sin(\theta)
}
\end{equation*}

\begin{itemize}
    \item We can take linear combinations of Fourier series. 
    \item We can shift a Fourier series by a constant.
    \item We can stretch a Fourier series by a constant.
\end{itemize}

\ex{Linear combinations of Fourier series}{
    We can take linear combinations of Fourier series by taking linear combinations of the Fourier coefficients. For example, if we have two functions $f$ and $g$ with Fourier coefficients $a_m^f, b_m^f$ and $a_m^g, b_m^g$ respectively, then the Fourier coefficients of the function $h = a f + b g$ are given by:
    \begin{align*}
    f(t) &= 1 + 2\text{sq}(t) \\
    &= 1 + \frac{8}{\pi} \left( \sin(t) + \frac{1}{3} \sin(3t) + \frac{1}{5} \sin(5t) + \cdots \right) \\
&= 1 + \frac{8}{\pi} \left( \sin\left( t + \frac{\pi}{2} \right) + \frac{1}{3} \sin\left( 3t + \frac{\pi}{2} \right) + \frac{1}{5} \sin\left( 5t + \frac{\pi}{2} \right) + \cdots \right) \\
&= 1 + \frac{8}{\pi} \left( \cos(t) + \frac{1}{3} \cos(3t) + \frac{1}{5} \cos(5t) + \cdots \right)
    \end{align*}
}

\ex{Shifting a Fourier series}{
        We can shift a Fourier series by a constant by using the angle addition formulas for sine and cosine. For example, if we have a function $f(t)$ with Fourier coefficients $a_m$ and $b_m$, then the Fourier coefficients of the function $g(t) = f(t - \frac{\pi}{2})$ are given by:
    \begin{align*}
    f(t) &= \text{sq}(t - \frac{\pi}{2}) \\
    &= \frac{4}{\pi} \left( \sin\left( t - \frac{\pi}{2} \right) + \frac{1}{3} \sin\left( 3t - \frac{3\pi}{2} \right) + \cdots \right) \\
    &= \frac{4}{\pi} \left( -\cos(t) + \frac{1}{3} \cos(3t) + \cdots \right)
    \end{align*}
}

\ex{Stretching a Fourier series}{
    We can stretch a Fourier series by a constant by using the angle addition formulas for sine and cosine. For example, if we have a function $f(t)$ with Fourier coefficients $a_m$ and $b_m$, then the Fourier coefficients of the function $g(t) = f(2t)$ are given by:
    \begin{align*}
    f(t) &= \text{sq}(2t) \\
    &= \frac{4}{\pi} \left( \sin(2t) + \frac{1}{3} \sin(6t) + \cdots \right)
    \end{align*}

    Then this function has period $\pi$, since the period of $\sin(2t)$ is $\pi$.
}

\subsection{Differentiating and integrating Fourier series}
We can differentiate a Fourier series term by term, since the Fourier series converges uniformly. For example, if we have a function $f(t)$ with Fourier coefficients $a_m$ and $b_m$, then the Fourier coefficients of the function $g(t) = f'(t)$ are given by:
\begin{align*}
f(t) &= \frac{a_0}{2} + \sum_{m=1}^{\infty} \left( a_m \cos(mt) + b_m \sin(mt) \right) \\
f'(t) &= \sum_{m=1}^{\infty} \left( -m a_m \sin(mt) + m b_m \cos(mt) \right)
\end{align*}

Similarly, we can integrate a Fourier series term by term. For example, if we have a function $f(t)$ with Fourier coefficients $a_m$ and $b_m$, then the Fourier coefficients of the function $g(t) = \int f(t) dt$ are given by:
\begin{align*}
f(t) &= \frac{a_0}{2} + \sum_{m=1}^{\infty} \left( a_m \cos(mt) + b_m \sin(mt) \right) \\
\int f(t) dt &= \frac{a_0}{2} t + \sum_{m=1}^{\infty} \left( \frac{a_m}{m} \sin(mt) - \frac{b_m}{m} \cos(mt) \right)
\end{align*}

\ex{Differentiating the square wave function}{
    We can differentiate the square wave function term by term to get:
    \begin{align*}
    f(t) &= \frac{4}{\pi} \left( \sin(t) + \frac{1}{3} \sin(3t) + \cdots \right) \\
    f'(t) &= \frac{4}{\pi} \left( \cos(t) + \cos(3t) + \cdots \right)
    \end{align*}

    Then we have that $f'(t)$ is a sum of cosine functions, which is a periodic function with period $2\pi$.
}