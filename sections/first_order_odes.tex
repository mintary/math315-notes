\chapter{Techniques for first-order ODEs}

\section{Separation of variables}

The standard form of a 1st order ODE is given by:
\begin{equation*}
\overline{R}(t) \dot x + \overline{P}(t) x = \overline{q} (t) \quad \overline{R}(t) \not = 0
\end{equation*}

If we let:
\begin{equation*}
p = \frac{\overline{P}(t)}{\overline{R}(t)} \quad , \quad q = \frac{\overline{q}(t)}{\overline{R}(t)}
\end{equation*}

Then we obtain the equation in the form:
\begin{equation*}
\boxed{\dot x + p(t) x = q(t)}
\end{equation*}

\dfn{Homogeneous equation}{
    An ODE is said to be \textbf{homogeneous} if $q(t) = 0$ for all $t$ in the interval of interest. 
}

Intuitively, the system evolves with no additional forcing or input beyond its initial conditions.

A homogeneous equation is also \textbf{separable} if it can be expressed in the form:
\begin{equation*}
\frac{\dot x}{x} = f(t)
\end{equation*}

\dfn{Separable equation}{
    An ODE is said to be \textbf{separable} if it can be expressed in the form:
    \begin{equation*}
    \frac{dx}{dt} = g(t) h(x)
    \end{equation*}
    where $g(t)$ is a function of $t$ only and $h(x)$ is a function of $x$ only.
}

We can easily solve separable equations by separating the variables and integrating both sides.
\ex{}{
    Solve the following ODE:
    \begin{equation*}
    \dot x + p(t) x = 0
    \end{equation*}
}
\sol{}{
    We can rewrite the equation as:
    \begin{equation*}
    \frac{\dot x}{x} = -p(t)
    \end{equation*}
    
    Then we separate the variables:
    \begin{align*}
    \frac{1}{x} \frac{dx}{dt} &= -p(t) \\
    \frac{1}{x} dx &= -p(t) dt
    \end{align*}
    
    Now we integrate both sides:
    \begin{align*}
    \int \frac{1}{x} dx &= \int -p(t) dt \\
    \ln |x| &= -\int p(t) dt + C \\
    |x| &= e^{-\int p(t) dt + C} \\
    x(t) &= \pm e^C e^{-\int p(t) dt} \\
    x(t) &= K e^{-\int p(t) dt} \quad , \quad K = \pm e^C
    \end{align*}
    
    Thus, the general solution to the ODE is:
    \begin{equation*}
    \boxed{x(t) = K e^{-\int p(t) dt}}
    \end{equation*}
}

\begin{mdframed}
\subsection{Steps to solve separable equations}
\begin{enumerate}
    \item Rewrite the ODE in the form $\frac{dx}{dt} = g(t) h(x)$.
    \item Separate the variables to obtain $\frac{1}{h(x)} dx = g(t) dt$.
    \item Integrate both sides: $\int \frac{1}{h(x)} dx = \int g(t) dt$.
    \item Solve for $x(t)$.
    \item (Optional) Apply any initial conditions to solve for constants of integration.
\end{enumerate}
\end{mdframed}

\dfn{Homogeneous solution}{
    In general, the solution to a homogeneous ODE is called the \textbf{homogeneous solution}.

    \begin{equation*}
    \boxed{
    X_h(t) = e^{-\int p(t) dt}
    }
    \end{equation*}

    Then:
    \begin{equation*}
    x(t) = K X_h(t)
    \end{equation*}

    for some constant $K$.
}

\ex{Newton's law of cooling}{
    The change in temperature $u$ over time $t$ is proportional to the difference in temperature between the object and its surroundings.

    Let $u_0$ be the initial temperature of the object and $T_{\text{ext}}$ be the external temperature. Let $k$ be the proportionality constant. Then:
    \begin{equation*}
    \dot u  = -k (u - T_{\text{ext}})
    \end{equation*}
}

\sol{}{
    We can rewrite the equation as:
    \begin{equation*}
    \dot u + k u = k T_{\text{ext}}
    \end{equation*}

    This is a first-order linear ODE. We can separate the variables:
    \begin{align*}
    \frac{du}{dt} &= -k (u - T_{\text{ext}}) \\
    \frac{1}{u - T_{\text{ext}}} du &= -k dt
    \end{align*}

    Then, we integrate both sides:
    \begin{align*}
    \int \frac{1}{u - T_{\text{ext}}} du &= \int -k dt \\
    \ln |u - T_{\text{ext}}| &= -kt + C \\
    |u - T_{\text{ext}}| &= e^{-kt + C} \\
    u - T_{\text{ext}} &= \pm e^C e^{-kt} \\
    u(t) &= \pm e^C e^{-kt} + T_{\text{ext}} \\
    u(t) &= K e^{-kt} + T_{\text{ext}} \quad , \quad K = \pm e^C
    \end{align*}

    To find $K$, we use the initial condition $u(0) = u_0$:
    \begin{align*}
    u(0) &= K e^{-k \cdot 0} + T_{\text{ext}} \\
    u_0 &= K + T_{\text{ext}} \\
    K &= u_0 - T_{\text{ext}}
    \end{align*}

    Therefore, the solution to the ODE is:
    \begin{equation*}
    \boxed{
    u(t) = (u_0 - T_{\text{ext}}) e^{-kt} + T_{\text{ext}}
    }
    \end{equation*}

    Intuitively, this can also be used to model warming as well as cooling. This depends on the sign of $(u_0 - T_{\text{ext}})$. 
}


\section{Integrating factors}

We are given an equation:
\begin{equation*}
\dot x + p(t) x = q(t)
\end{equation*}


The method of integrating factors can be thought of intuitively as "undoing" the product rule of differentiation. Specifically, we would like to transform the left-hand side of the equation into the derivative of a product of two functions $u(t)$ and $x(t)$:
\begin{equation*}
\frac{d}{dt} \left[ u(t) x(t) \right] = u(t) \dot x(t) + \dot u(t) x(t)
\end{equation*}

Looking at our original equation, we'd like to have some $\mu(t)$ to multiply throughout the equation such that:
\begin{align*}
\mu(t) \dot x + \mu(t) p(t) x &= u(t) \dot x + \dot u(t) x \\
&= \frac{d}{dt} \left[ u(t) x(t) \right]
\end{align*}

This means we need to have:
\[
\begin{cases}
\mu(t) = u(t) \\
\mu(t) p(t) = \dot u(t)
\end{cases}
\]

Solving for $\mu(t)$, we have:
\begin{align*}
\dot u = up(t) \\
\frac{du}{dt} = up(t) \\
\frac{1}{u} du = p(t) dt \\
\int \frac{1}{u} du = \int p(t) dt \\
\ln |u| = \int p(t) dt + C \\
u(t) = e^{\int p(t) dt + C} = e^C e^{\int p(t) dt}
\end{align*}

We can ignore the constant as it will cancel out later.

\dfn{Integrating factor}
{
    The \textbf{integrating factor} for the first-order linear ODE $\dot x + p(t) x = q(t)$ is given by:
    \begin{equation*}
    \boxed{
    \mu(t) = e^{\int p(t) dt}
    }
    \end{equation*}

    Also, this is the reciprocal of the homogeneous solution $X_h(t)$ from separation of variables:
    \begin{equation*}
    \mu(t) = X^{-1}_h(t)
    \end{equation*}
}

\begin{mdframed}
\subsection{Steps to solve using integrating factors}
\begin{enumerate}
    \item Start with the standard form of the ODE: $\dot x + p(t) x = q(t)$.
    \item Compute the integrating factor: $\mu(t) = e^{\int p(t) dt}$.
    \item Multiply both sides of the ODE by the integrating factor $\mu(t)$.
    \item Recognize that the left-hand side is now the derivative of the product:
    \[
        \frac{d}{dt} \left[ \mu(t) x(t) \right] = \mu(t) q(t)
    \]
    \item Integrate both sides with respect to $t$:
    \[
        \mu(t) x(t) = \int \mu(t) q(t) dt + C
    \]
    \item Solve for $x(t)$:
    \[
        x(t) = \frac{1}{\mu(t)} \left( \int \mu(t) q(t) dt + C \right)
    \]
    \item (Optional) Apply any initial conditions to solve for the constant $C$.
\end{enumerate}
\end{mdframed}

\ex{}{
    Solve the initial value problem:
    \begin{equation*}
    t\dot x + 2x = t^2 - t + 1, \quad x(1) = \frac{1}{2}
    \end{equation*}
}
\sol{}{
    \begin{enumerate}
        \item Standard form: $\dot x + \frac{2}{t} x = t - 1 + \frac{1}{t}$
        \item Integrating factor:
        \begin{align*}
        \mu(t) &= e^{\int \frac{2}{t} dt} \\
        &= e^{2 \ln |t|} \\
        &= t^2
        \end{align*}
        \item Multiply through by $\mu(t)$:
        \begin{equation*}
        t^2 \dot x + 2t x = t^3 - t^2 + t
        \end{equation*}
        \item Left-hand side as derivative:
        \begin{equation*}
        \frac{d}{dt} \left[ t^2 x \right] = t^3 - t^2 + t
        \end{equation*}
        \item Integrate both sides:
        \begin{align*}
        t^2 x &= \int (t^3 - t^2 + t) dt + C \\
        &= \frac{t^4}{4} - \frac{t^3}{3} + \frac{t^2}{2} + C
        \end{align*}
        \item Solve for $x(t)$:
        \begin{equation*}
        x(t) = \frac{1}{t^2} \left( \frac{t^4}{4} - \frac{t^3}{3} + \frac{t^2}{2} + C \right) = \frac{t^2}{4} - \frac{t}{3} + \frac{1}{2} + \frac{C}{t^2}
        \end{equation*}
        \item Apply initial condition $x(1) = \frac{1}{2}$:
        \begin{align*}
        \frac{1}{2} &= \frac{1}{4} - \frac{1}{3} + \frac{1}{2} + C \\
        C &= \frac{1}{12}
        \end{align*}
    \end{enumerate}
    \begin{equation*}
    \boxed{
    x(t) = \frac{t^2}{4} - \frac{t}{3} + \frac{1}{2} + \frac{1}{12t^2}
    }
    \end{equation*}
}

\section{Substitution technique}

Consider the differential equation:
\begin{equation*}
\dot y = F(\frac{y}{x})
\end{equation*}

We can eliminate the $x$ in the denominator to have a function of a single variable by substituting:
\begin{equation*}
v = \frac{y}{x} \implies y = vx
\end{equation*}

Then rearranging:
\begin{align*}
\dot y &= F(v) \\
v + x \dot v &= F(v)  \\
x \dot v &= F(v) - v \\
x \cdot \frac{dv}{dx} &= F(v) - v \\
\frac{dv}{dx} &= \frac{F(v) - v}{x} \\
\frac{dv}{F(v) - v} &= \frac{dx}{x} \quad \text{(separable equation)}
\end{align*}

\begin{mdframed}
\subsection{Steps to solve using substitution technique}
\begin{enumerate}
    \item Identify the substitution: $v = \frac{y}{x}$, which implies $y = vx$.
    \item Differentiate $y$ with respect to $x$ using the product rule:
    \[
        \frac{dy}{dx} = v + x \frac{dv}{dx}
    \]
    \item Substitute $\frac{dy}{dx}$ and $y$ into the original ODE to express it in terms of $v$ and $x$.
    \item Rearrange the equation to isolate $\frac{dv}{dx}$.
    \item Separate the variables to obtain an equation of the form:
    \[
        \frac{dv}{F(v) - v} = \frac{dx}{x}
    \]
    \item Integrate both sides to find $v(x)$.
    \item Substitute back to find $y(x)$ using $y = vx$.
    \item (Optional) Apply any initial conditions to solve for constants of integration.
\end{enumerate}
\end{mdframed}

We can now solve this separable equation for $v(x)$, then substitute back to find $y(x)$.
\ex{Tutorial 3 problem 1}{
    Solve the initial value problem:
    \[
    \begin{cases*}
    \dot y = \frac{x^2 + 3y^2}{2xy} \\
    y(-2) = 6
    \end{cases*}
    \]
}

\sol{}{
    We rewrite so we can make some substitution for $u = \frac{y}{x}$:
    \begin{align*}
    \dot y &= \frac{x^2 + 3y^2}{2xy} \\
    &= \frac{x^2}{2xy} + \frac{3y^2}{2xy} \\
    &= \frac{x}{2y} + \frac{3y}{2x} \\
    &= \frac{1}{2 \cdot \frac{y}{x}} + \frac{3}{2} \cdot \frac{y}{x} \\
    &= \frac{1}{2u} + \frac{3}{2} u
    \end{align*}

    Recall that $u = \frac{y}{x} \implies y = ux$. Then $\frac{dy}{dx} = u + x \frac{du}{dx}$. Substituting this in:
    \begin{align*}
    u + x \frac{du}{dx} &= \frac{1}{2u} + \frac{3}{2} u \\
    x \frac{du}{dx} &= \frac{1}{2u} + \frac{3}{2} u - u \\
    x \ du = \left( \frac{1}{2u} + \frac{1}{2} u \right) dx \\
    \frac{1}{\frac{1}{2u} + \frac{1}{2} u} \ du = \frac{dx}{x} \\
    \end{align*}

    Note that this is effectively transforming the original equation into the separable form $\frac{dv}{F(v) - v} = \frac{dx}{x}$.

    Now we integrate by partial fractions:
    \begin{align*}
    \int \frac{1}{\frac{1}{2u} + \frac{1}{2} u} \ du &= \int \frac{dx}{x} \\
    \int \frac{2u}{1 + u^2} \ du &= \ln |x| + C \\
    \int \frac{1}{1 + u^2} \ d(u^2) &= \ln |x| + C \\
    \ln |1 + u^2| &= \ln |x| + C \\
    1 + u^2 &= K |x| \ , \quad K = e^C
    1 + u^2 &= Bx \, \quad B = \pm K
    \end{align*}

    Then substituting back for $u = \frac{y}{x}$:
    \begin{align*}
    1 + \left( \frac{y}{x} \right)^2 &= Bx \\
    1 + \frac{y^2}{x^2} &= Bx \\
    y^2 &= Bx^3 - x^2
    \end{align*}

    Rearranging gives the general solution:
    \begin{equation*}
    y^2 = Bx^3 - x^2
    \end{equation*}

    We can solve for $y$:
    \begin{equation*}
    \boxed{y = \pm \sqrt{Bx^3 - x^2}}
    \end{equation*}

    Now using the initial condition $y(-2) = 6$ to solve for $B$:
    \begin{align*}
    6 &= \pm \sqrt{B(-2)^3 - (-2)^2} \\
    36 &= -8B - 4 \\
    40 &= -8B \\
    B &= -5
    \end{align*}

    We must choose the positive root since $y(-2) = 6 > 0$. Thus the particular solution is:
    \begin{equation*}
    \boxed{y = \sqrt{-5x^3 - x^2}}
    \end{equation*}

    The domain is $-5x^3 - x^2 \geq 0 \implies x^2(-5x - 1) \geq 0 \implies x \leq -\frac{1}{5}$.
}

\section{Exact equations}

We have solely dealt with first-order linear and separable differential equations so far. However, some ODEs will have functions of both $x$ and $y$ that prevent separation.

\ex{}{
    Solve the following ODE:
    \begin{equation*}
    (2xy - 9x^2) + (2y + x^2 + 1) \frac{dy}{dx} = 0
    \end{equation*}
}

\sol{}{
Now consider some function $\Psi(x, y)$ (don't worry about how we got it yet) such that:
\begin{equation*}
\Psi(x, y) = y^2 + (x^2 + 1)y - 3x^3
\end{equation*}

If we compute the partial derivatives, we find:
\begin{align*}
\Psi_x &= 2xy - 9x^2 \\
\Psi_y &= 2y + x^2 + 1
\end{align*}

These expressions appear in our original equation. Using the chain rule for partial derivatives:
\begin{align*}
    \frac{d}{dx} \Psi(x, y(x)) &= \Psi_x \frac{dx}{dx} + \Psi_y \frac{dy}{dx} \\
    \frac{d}{dx} \Psi(x, y(x)) &= \Psi_x + \Psi_y \frac{dy}{dx}
\end{align*}

Then we can rewrite the original ODE as:
\begin{equation*}
\Psi_x + \Psi_y \frac{dy}{dx} = 0
\end{equation*}

Thus, we have:
\begin{equation*}
\frac{d}{dx} \Psi(x, y) = 0
\end{equation*}

Integrating both sides with respect to $x$ gives:
\begin{equation*}
\Psi(x, y) = C
\end{equation*}

So the general solution to the ODE is:
\begin{equation*}
\boxed{
y^2 + (x^2 + 1)y - 3x^3 = C
}
\end{equation*}
}

We are therefore concerned with obtaining a method to find such a function $\Psi(x, y)$ for a given ODE.

\dfn{Exact equation}{
    An ODE of the form:
    \begin{equation*}
    M(x, y) + N(x, y) \frac{dy}{dx} = 0
    \end{equation*}
    is said to be an \textbf{exact equation} if there exists a function $\Psi(x, y)$ such that:
    \begin{align*}
    \Psi_x &= M(x, y) \\
    \Psi_y &= N(x, y)
    \end{align*}

    If an ODE is exact, we have:
    \begin{equation*}
    \boxed{
    \frac{d}{dx} \Psi(x, y) = \Psi_x + \Psi_y \frac{dy}{dx} = 0
    }
    \end{equation*}

    Then the solution to the ODE is given implicitly by:
    \begin{equation*}
    \boxed{
    \Psi(x, y) = C
    }
    \end{equation*}
}

If $\Psi(x, y)$ is continuously differentiable, then we have that:
\begin{equation*}
(\Psi_x)_y = (\Psi_y)_x
\end{equation*}

Then the equation is only exact if:
\begin{equation*}
\boxed{
M_y = N_x
}
\end{equation*}


\begin{mdframed}
\subsection{Steps to solve exact equations}
\begin{enumerate}
    \item Verify that the equation is exact by checking if $M_y = N_x$.
    \item Integrate $M(x, y)$ with respect to $x$ to find $\Psi(x, y)$ up to a function of $y$:
    \begin{equation*}
    \Psi(x, y) = \int M(x, y) \, dx + h(y)
    \end{equation*}
    \item Differentiate $\Psi(x, y)$ with respect to $y$ and set it equal to $N(x, y)$ to solve for $h(y)$:
    \begin{equation*}
    \frac{\partial \Psi}{\partial y} = N(x, y)
    \end{equation*}
    \item Substitute $h(y)$ back into $\Psi(x, y)$.
    \item (Optional) Apply the initial condition if given back into:
    \begin{equation*}
    \Psi(x, y) = C
    \end{equation*}
\end{enumerate}
\end{mdframed}

\nt{
    If it is easier, you can also integrate $N(x, y)$ with respect to $y$ first, then differentiate with respect to $x$ to find the function of $x$.
}

\ex{}{
    Solve the following initial value problem:
    \begin{equation*}
    2xy - 9x^2 + (2y + x^2 + 1) \frac{dy}{dx} = 0, \quad y(0) = -3
    \end{equation*}
}
\sol{}{
    \begin{enumerate}
        \item Find and verify $M_y$ and $N_x$:
        \begin{align*}
        M(x, y) &= 2xy - 9x^2 \\
        N(x, y) &= 2y + x^2 + 1 \\
        M_y(x, y) &= 2x \\
        N_x(x, y) &= 2x
        \end{align*}
        Since $M_y = N_x$, the equation is exact.
        \item Integrate $M(x, y)$ with respect to $x$:
        \begin{align*}
        \Psi(x, y) &= \int (2xy - 9x^2) \, dx + h(y) \\
        &= x^2 y - 3x^3 + h(y)
        \end{align*}
        \item Differentiate $\Psi(x, y)$ with respect to $y$ and set equal to $N(x, y)$:
        \begin{align*}
        \frac{\partial \Psi}{\partial y} &= x^2 + h'(y) \\
        x^2 + h'(y) &= 2y + x^2 + 1 \\
        h'(y) &= 2y + 1
        \end{align*}
        Integrating gives:
        \begin{equation*}
        h(y) = y^2 + y + K
        \end{equation*}
        \item Substitute $h(y)$ back into $\Psi(x, y)$:
        \begin{equation*}
        \Psi(x, y) = x^2 y - 3x^3 + y^2 + y + K
        \end{equation*}
        \item Apply the initial condition $y(0) = -3$:
        \begin{align*}
        \Psi(0, -3) &= 0^2 \cdot (-3) - 3 \cdot 0^3 + (-3)^2 + (-3) + K \\
        &= 0 + 0 + 9 - 3 + K \\
        &= 6 + K
        \end{align*}
        Setting this equal to $C$, we have:
        \begin{equation*}
        C = 6 + K
        \end{equation*}
        Thus, the particular solution is:
        \begin{equation*}
        \boxed{
        x^2 y - 3x^3 + y^2 + y = 6
        }
        \end{equation*}
        \item We could also solve for $y$ explicitly using the quadratic formula:
        \begin{align*}
        y^2 + (x^2 + 1)y + (-3x^3 - 6) &= 0 \\
        y &= \frac{-(x^2 + 1) \pm \sqrt{(x^2 + 1)^2 - 4(-3x^3 - 6)}}{2} \\
        y &= \frac{-(x^2 + 1) \pm \sqrt{x^4 + 2x^2 + 1 + 12x^3 + 24}}{2} \\
        y &= \frac{-(x^2 + 1) \pm \sqrt{x^4 + 12x^3 + 2x^2 + 25}}{2}
        \end{align*}
        We must now choose the correct sign for the initial condition $y(0) = -3$, since $-3$ is less than $\frac{-(0^2 + 1)}{2} = -\frac{1}{2}$. Thus we choose the negative root:
        \begin{equation*}
        \boxed{
        y = \frac{-(x^2 + 1) - \sqrt{x^4 + 12x^3 + 2x^2 + 25}}{2}
        }
        \end{equation*}

    \end{enumerate}
}