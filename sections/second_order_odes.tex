\chapter{Techniques for solving second-order ODEs}

\section{Classical mechanics: mass-spring system}

Consider a mass $m$ attached to a spring with spring constant $k$. It is also subject to an external force $F_{\text{ext}}$.

\begin{center}
\includegraphics[width=0.4\textwidth]{images/second_order_odes/mass_spring_system.png}
\end{center}


\vspace{0.2cm}

Note that velocity is the first derivative of position with respect to time, denoted $\dot x(t)$, and acceleration is the second derivative of position with respect to time, denoted $\ddot x(t)$.

Hooke's law gives the force exerted by the spring:
\begin{equation*}
F_{\text{spring}} = -k x, \quad k > 0
\end{equation*}

Then, by Newton's second law, the total force on the mass is equal to its mass times its acceleration:
\begin{equation*}
m \ddot x = F_{\text{spring}} + F_{\text{ext}}
\end{equation*}
Rearranging gives the second-order ODE:
\begin{equation*}
m \ddot x + k x = F_{\text{ext}}
\end{equation*}

We could also apply a damping force (such as friction) proportional to velocity, with damping coefficient $b$. We call this:
\begin{equation*}
F_{\text{damping}} = -b \dot x, \quad b > 0
\end{equation*}

Then the ODE becomes:
\begin{equation*}
m \ddot x + b \dot x + k x = F_{\text{ext}}
\end{equation*}

\dfn{Linear ODE}{
    A linear ODE is one of the form:
    \begin{equation*}
    a_m x^{(m)} + a_{m-1} x^{(m-1)} + \cdots + a_1 \dot x + a_0 x = q(t)
    \end{equation*}
    where $a_0, a_1, \ldots, a_m$ are coefficients that may depend on $t$ but not on $x$ or its derivatives, and $q(t)$ is a function of $t$.

    We call this a $m$-th order linear ODE.
}

In our mass-spring system, we will have two initial conditions.
\begin{enumerate}
    \item Initial position: $x(0) = x_0$
    \item Initial velocity: $\dot x(0) = v_0$
\end{enumerate}

\ex{A special case}{
    Consider a mass-spring system where $b = 0$ (no damping) and there is no external force ($F_{\text{ext}} = 0$). Solve the ODE:
    \begin{equation*}
    m \ddot x + k x = 0
    \end{equation*}
    with initial conditions $x(0) = x_0$ and $\dot x(0) = v_0$.
}

\sol{}{
    Without knowing any specific methods, we rearrange the equation:
    \begin{equation*}
    \ddot x = -\frac{k}{m} x
    \end{equation*}
    So we need some function whose second derivative is proportional to the negative of the function itself. We know that sine and cosine functions have this property.

    Try $x = \sin(\omega t)$, and differentiate several times:
    \begin{align*}
    \dot x &= \omega \cos(\omega t) \\
    \ddot x &= -\omega^2 \sin(\omega t)
    \end{align*}

    Then substituting into the ODE gives:
    \begin{equation*}
    -m \omega^2 \sin(\omega t) + k \sin(\omega t) = 0
    \end{equation*}

    \begin{equation*}
    \boxed{
        x(t) = \sin(\omega t), \quad \text{where } \omega = \sqrt{\frac{k}{m}}
    }
    \end{equation*}

    This is only a particular solution (you could try with cosine as well, and the general solution is a linear combination of both). Intuitively, the mass will oscillate back and forth indefinitely.
}

\thm{Superposition principle (applied to second-order linear ODEs)}{
    For a linear ODE, if $x_1(t)$ and $x_2(t)$ are solutions to the homogeneous equation (i.e., when $q(t) = 0$), then any linear combination of these solutions is also a solution. That is:
    \begin{equation*}
    x(t) = C_1 x_1(t) + C_2 x_2(t)
    \end{equation*}
    is also a solution for any constants $C_1$ and $C_2$. The functions $x_1(t)$ and $x_2(t)$ are called the \textbf{modes} of the system.
}
\pf{Proof}{
    Let $x_1(t)$ and $x_2(t)$ be solutions to the homogeneous equation:
    \begin{equation*}
    a_2 \ddot x + a_1 \dot x + a_0 x = 0
    \end{equation*}

    Then, substituting the potential solution $x(t) = C_1 x_1(t) + C_2 x_2(t)$:
    \begin{align*}
    a_2 \ddot x + a_1 \dot x + a_0 x &= a_2 \left( C_1 \ddot x_1 + C_2 \ddot x_2 \right) + a_1 \left( C_1 \dot x_1 + C_2 \dot x_2 \right) + a_0 \left( C_1 x_1 + C_2 x_2 \right) \\
    &= C_1 \left( a_2 \ddot x_1 + a_1 \dot x_1 + a_0 x_1 \right) + C_2 \left( a_2 \ddot x_2 + a_1 \dot x_2 + a_0 x_2 \right) \\
    &= C_1 \cdot 0 + C_2 \cdot 0 = 0
    \end{align*}
    Thus, $x(t) = C_1 x_1(t) + C_2 x_2(t)$ is also a solution.
}

\ex{General solution}{
    Find the general solution to the ODE:
    \begin{equation*}
    m \ddot x + k x = 0
    \end{equation*}
}
\sol{}{
    From the previous example, we know that both $\sin(\omega t)$ and $\cos(\omega t)$ are solutions, where $\omega = \sqrt{\frac{k}{m}}$.

    Therefore, by the superposition principle, the general solution is:
    \begin{equation*}
    \boxed{
        x(t) = C_1 \cos(\omega t) + C_2 \sin(\omega t)
    }
    \end{equation*}
    where $C_1$ and $C_2$ are constants determined by the initial conditions.
}

\nt{
    The functions $\sin(\omega t)$ and $\cos(\omega t)$ are linearly independent because to transform one into another, you would need to add some factor in the argument (i.e., phase shift). This is not a scalar multiple transformation, so no linear transformation exists between them.
}

\section{Equations with constant coefficients}

\subsection{Deriving the characteristic equation}

We make the ansatz (educated guess) that the solution is of the form:
\begin{equation*}
x(t) = c e^{r t}
\end{equation*}
where $r$ is a constant to be determined. This guess is motivated by the fact that exponentials have the property that their derivatives are proportional to themselves (remember how we worked with the $\sin$ and $\cos$ functions earlier).

We want to attempt to form a linear combination of such solutions.

Recalling our equation of the form:
\begin{equation*}
m \ddot x + b \dot x + k x = 0
\end{equation*}

Substituting our ansatz into the ODE gives:
\begin{equation*}
m r^2 c e^{r t} + b r c e^{r t} + k c e^{r t} = 0
\end{equation*}

The trivial solution is $c = 0$, but we are interested in non-trivial solutions where $c \neq 0$. Dividing both sides by $c e^{r t}$ (which is never zero) gives the characteristic equation:
\begin{equation*}
m r^2 + b r + k = 0
\end{equation*}

This is the \textbf{characteristic equation} of the ODE. It can be observed that we will obtain two roots, or two \textbf{modes}. This makes sense - we expect two linearly independent solutions for a second-order ODE, with the general solution being a linear combination of these two solutions.

\subsection{Distinct real roots}

\ex{}{
    Find the general solution to the ODE:
    \begin{equation*}
        \ddot x + 5 \dot x + 4x  = 0
    \end{equation*}

    Then, find the specific solution satisfying the initial conditions $x(0) = 2$ and $\dot x(0) = -5$.
}
\sol{}{
    We write this as 
    \begin{equation*}
        1 \cdot \ddot x + 5 \cdot \dot x + 4 \cdot x = 0
    \end{equation*}
    so that we can identify $m = 1$, $b = 5$, and $k = 4$.
    Then, the characteristic equation is:
    \begin{equation*}
        P(s) = 1 \cdot s^2 + 5 \cdot s + 4 = 0
    \end{equation*}
    Solving for the roots, we have:
    \begin{equation*}
        r^2 + 5r + 4 = 0
    \end{equation*}
    Factoring gives:
    \begin{equation*}
        (r + 4)(r + 1) = 0
    \end{equation*}
    Thus, the roots are $r_1 = -4$ and $r_2 = -1$.

    Therefore, the general solution is:
    \begin{equation*}
        \boxed{
            x(t) = C_1 e^{-4t} + C_2 e^{-t}
        }
    \end{equation*}
    where $C_1$ and $C_2$ are constants determined by initial conditions.

    To find the specific solution, we apply the initial conditions:
    \begin{align*}
        x(0) &= C_1 e^{0} + C_2 e^{0} = C_1 + C_2 = 2 \quad (1) \\
        \dot x(t) &= -4 C_1 e^{-4t} - C_2 e^{-t} \\
        \dot x(0) &= -4 C_1 e^{0} - C_2 e^{0} = -4 C_1 - C_2 = -5 \quad (2)
    \end{align*}
    Solving equations (1) and (2):
    From (1): $C_2 = 2 - C_1$.
    Substituting into (2):
    \begin{align*}
        -4 C_1 - (2 - C_1) &= -5 \\
        -4 C_1 - 2 + C_1 &= -5 \\
        -3 C_1 - 2 &= -5 \\
        -3 C_1 &= -3 \\
        C_1 &= 1
    \end{align*}
    Then from (1):
    \begin{equation*}
        C_2 = 2 - 1 = 1
    \end{equation*}
    Thus, the specific solution is:
    \begin{equation*}
        \boxed{
            x(t) = e^{-4t} + e^{-t}
        }
    \end{equation*}
}

With an equation of the form $m \ddot x + b \dot x + kx = 0$, we can summarize the types of solutions based on the discriminant $D = b^2 - 4mk$:
\begin{itemize}
    \item \textbf{Overdamped} ($D > 0$): Two distinct real roots, leading to solutions that decay exponentially without oscillating.
    \[
        x(t) = C_1 e^{r_1 t} + C_2 e^{r_2 t}
    \]
    \item \textbf{Critically damped} ($D = 0$): One repeated real root, leading to solutions that decay to zero as quickly as possible without oscillating.
    \[
        x(t) = C_1 e^{r t} + C_2 t e^{r t}
    \]
    \item \textbf{Underdamped} ($D < 0$): Complex conjugate roots, leading to \textbf{oscillatory} solutions that decay exponentially.
\end{itemize}

\subsection{Review of complex numbers}
\nt{
    The lectures started from the introduction of $i^2 = -1$. These notes skip ahead to the parts more relevant to ODEs.
}

\dfn{Argument of a complex number}{
    The \textbf{argument} of a complex number $z = a + bi$ is the angle $\theta$ formed with the positive real axis in the complex plane. It is denoted as $\arg(z)$.

    The argument can be calculated using the arctangent function:
    \begin{equation*}
    \theta = \tan^{-1}\left(\frac{b}{a}\right)
    \end{equation*}
}

Some useful properties of complex numbers and their arguments:
\begin{align*}
    | z_1 z_2 | &= |z_1| |z_2| \\
    \arg(z_1 z_2) &= \arg(z_1) + \arg(z_2) \\
    \arg(z^n) &= n \arg(z) \\
\end{align*}


We can parametrize a complex number on the unit circle by letting $t = \arg(z)$:
\begin{equation*}
z = \cos t + i \sin t
\end{equation*}

Then, differentiating:
\begin{equation*}
\dot z = -\sin t + i \cos t
\end{equation*}

This then implies a useful identity:
\begin{equation*}
\boxed{
    \dot z = i z
}
\end{equation*}

Solving this ODE gives:
\begin{equation*}
z(t) = z(0) e^{i t}
\end{equation*}

\textbf{Euler's identity} states that:
\begin{equation*}
\boxed{
    e^{i t} = \cos t + i \sin t
}
\end{equation*}

This is useful because it allows us to express trigonometric functions in terms of exponentials.

Another interesting identity arises when we set $t = \pi$:
\begin{equation*}
    e^{i \pi} + 1 = 0
\end{equation*}

Now, consider a complex number $z = a + bi$. We have that:
\begin{align*}
\dot z &= (a+bi) z \\
z(t) &= z(0) e^{(a+bi)t} \\
&= z(0) e^{at} (\cos bt + i \sin bt)
\end{align*}

where $e^{at}$ represents the magnitude (scaling) and $\cos bt + i \sin bt$ represents the rotation in the complex plane (argument).

Another useful property of complex numbers:
\begin{equation*}
\boxed{
    \text{Re}(z) = \frac{1}{2} (z + \overline{z}) 
}
\end{equation*}

\begin{equation*}
\boxed{
    \text{Im}(z) = \frac{1}{2i} (z - \overline{z})
}
\end{equation*}

where $\overline{z} = a - bi$ is the complex conjugate of $z = a + bi$.

Then we define $\cos$ and $\sin$ in terms of exponentials:
\begin{equation*}
\boxed{
    \cos t = \frac{1}{2}\left( e^{i t} + e^{-i t} \right)
}
\end{equation*}
\begin{equation*}
\boxed{
    \sin t = \frac{1}{2i} \left( e^{i t} - e^{-i t} \right)
}
\end{equation*}

\subsection{Complex roots}

\thm{}{
    If $z$ is a complex-valued solution to $m \ddot z + b \dot z + k z = 0$, then both the real part $\text{Re}(z)$ and the imaginary part $\text{Im}(z)$ are also solutions to the ODE.
}

\pf{Proof}{
This should not be surprising. If we write $z = u(t) + iv(t)$ where $u(t) = \text{Re}(z)$ and $v(t) = \text{Im}(z)$, then substituting into the ODE gives:
\begin{align*}
m \ddot z + b \dot z + k z &= m (\ddot u + i \ddot v) + b (\dot u + i \dot v) + k (u + i v) \\
&= (m \ddot u + b \dot u + k u) + i (m \ddot v + b \dot v + k v)
\end{align*}

We basically get two separate equations, corresponding to the real and imaginary parts:
\begin{align*}
m \ddot u + b \dot u + k u &= 0 \\
m \ddot v + b \dot v + k v &= 0
\end{align*}

Since $z$ is a solution, both parts must equal zero. Thus, both $\text{Re}(z)$ and $\text{Im}(z)$ are solutions to the ODE.

}

We can observe that an oscillatory solution arises when the roots of the characteristic equation are complex conjugates. To derive this, we use the second-order ODE:
\begin{equation*}
r_1 = a + bi, \quad r_2 = a - bi
\end{equation*}

Extracting the real and imaginary parts gives us two linearly independent solutions:
\begin{align*}
x_1(t) &= e^{(a + bi)t} = e^{a t} \cos(b t) \\
x_2(t) &= e^{(a - bi)t} = e^{a t} \sin(b t)
\end{align*}

So the general solution is a linear combination of these two solutions (where $a$ determines the exponential decay or growth - the real part, and $b$ determines the frequency of oscillation - the imaginary part):
\begin{equation*}
\boxed{
    x(t) = C_1 e^{a t} \cos(b t) + C_2 e^{a t} \sin(b t)
}
\end{equation*}

We can also represent this as a phase shifted cosine function:
\begin{equation*}
\boxed{
    x(t) = A e^{a t} \cos(b t - \phi)
}
\end{equation*}

where $A$ is the amplitude and $\phi$ is the phase shift, which can be determined from the constants $C_1$ and $C_2$. Formulas for $A$ and $\phi$ are:
\begin{equation*}
\boxed{
A = \sqrt{C_1^2 + C_2^2}
}
\end{equation*}

\begin{equation*}
\boxed{
\phi = \tan^{-1}\left( \frac{C_2}{C_1} \right)
}
\end{equation*}

The identity used to get this phase shift:
\begin{align*}
\cos(\alpha - \beta) &= \cos \alpha \cos \beta + \sin \alpha \sin \beta \\
&= \frac{C_1}{A} \cos b t + \frac{C_2}{A} \sin b t
\end{align*}

\ex{}{
    Find the general solution to the ODE:
    \begin{equation*}
        \ddot x + 4 \dot x + 5x = 0
    \end{equation*}
}

\sol{}{
    We write this as 
    \begin{equation*}
        1 \cdot \ddot x + 4 \cdot \dot x + 5 \cdot x = 0
    \end{equation*}
    so that we can identify $m = 1$, $b = 4$, and $k = 5$.
    Then, the characteristic equation is:
    \begin{equation*}
        P(s) = 1 \cdot s^2 + 4 \cdot s + 5 = 0
    \end{equation*}
    Solving for the roots, we have:
    \begin{equation*}
        r^2 + 4r + 5 = 0
    \end{equation*}
    Using the quadratic formula:
    \begin{equation*}
        r = \frac{-4 \pm \sqrt{16 - 20}}{2} = -2 \pm i
    \end{equation*}
    Thus, the roots are $r_1 = -2 + i$ and $r_2 = -2 - i$.
    Therefore, the general solution is:
    \begin{equation*}
        \boxed{
            x(t) = e^{-2t} (C_1 \cos t + C_2 \sin t)
        }
    \end{equation*}
    We can also express this as a phase shifted cosine function:
    \begin{equation*}
        \boxed{
            x(t) = A e^{-2t} \cos(t - \phi)
        }
    \end{equation*}
}

\subsection{Repeated roots}

Let $P(D)$ be the polynomial differential operator associated with a linear ODE, where $Q$ is a polynomial with no repeated roots.
If $r$ is a root of multiplicity $m$ of the characteristic polynomial $P(\lambda)$,
then
\[
P(\lambda) = (\lambda - r)^m Q(\lambda)
\quad\Rightarrow\quad
P(D) = (D - r)^m Q(D).
\]

Since $Q(r)\neq 0$:
\[
(D-r)^m y = 0.
\]

Solving successively,
\begin{align*}
(D-r) y &= 0 \\
Dy - r y &= 0 \\
\frac{dy}{dt} &= r y \\
y &= e^{rt}
\end{align*}
and each additional application of $(D-r)^{-1}$ introduces a factor of $t$.
Hence the $m$ linearly independent solutions are
\[
e^{rt},\; t e^{rt},\; t^2 e^{rt},\; \ldots,\; t^{m-1} e^{rt}.
\]

\ex{}{
    Find the general solution to the ODE:
    \begin{equation*}
        \ddot x - 4 \dot x + 4x = 0
    \end{equation*}
}

\sol{}{
    We write this as 
    \begin{equation*}
        1 \cdot \ddot x - 4 \cdot \dot x + 4 \cdot x = 0
    \end{equation*}
    so that we can identify $m = 1$, $b = -4$, and $k = 4$.
    Then, the characteristic equation is:
    \begin{equation*}
        P(s) = 1 \cdot s^2 - 4 \cdot s + 4 = 0
    \end{equation*}
    Solving for the roots, we have:
    \begin{equation*}
        r^2 - 4r + 4 = 0
    \end{equation*}
    Factoring gives:
    \begin{equation*}
        (r - 2)^2 = 0
    \end{equation*}
    Thus, there is a repeated root at $r = 2$.

    Therefore, the general solution is:
    \begin{equation*}
        \boxed{
            x(t) = C_1 e^{2t} + C_2 t e^{2t}
        }
    \end{equation*}
}

\section{Non-homogeneous equations}

We were previous using $m\ddot x + b \ddot x + kx = 0$ to represent a situation where a mass is attached to a spring with no external damping force.

If we introduce an external force $F_{\text{ext}}$, then the ODE becomes:
\begin{equation*}
m \ddot x + b \dot x + k x = F_{\text{ext}}
\end{equation*}

We know the solution to the homogeneous equation. Also, if $b=0$, we can define the \textbf{natural frequency} as:
\begin{equation*}
w_m = \sqrt{\frac{k}{m}}
\end{equation*}

such that the solution is:
\begin{equation*}
x(t) = C_1 \cos(w_m t) + C_2 \sin(w_m t) \quad \text{from the characteristic equation solved via the quadratic formula}
\end{equation*}

And if $b \neq 0$, we can define the \textbf{damped frequency} as:
\begin{equation*}
w_d = \sqrt{\frac{k}{m} - \frac{b^2}{4 m^2}} \quad \text{from the characteristic equation solved via the quadratic formula}
\end{equation*}

such that the solution is:
\begin{equation*}
x(t) = e^{-\frac{b}{2m} t} (C_1 \cos(w_d t) + C_2 \sin(w_d t))
\end{equation*}

We standardize the homogeneous solution as $x_h(t)$:
\begin{equation*}
\ddot x + \frac{b}{m} \dot x + w_m^2 x = 0
\end{equation*}

where $w_m$ is the natural frequency.

\thm{}{
    If $x_p(t)$ is a particular solution to the non-homogeneous equation $m \ddot x + b \dot x + k x = F_{\text{ext}}$, then the general solution is given by:
    \begin{equation*}
    x(t) = x_h(t) + x_p(t)
    \end{equation*}
    where $x_h(t)$ is the general solution to the corresponding homogeneous equation.
}
\pf{Proof}{
    Let $x_h(t)$ be the general solution to the homogeneous equation:
    \begin{equation*}
    m \ddot x + b \dot x + k x = 0
    \end{equation*}

    Then, we can write the general solution to the non-homogeneous equation as:
    \begin{equation*}
    x(t) = x_h(t) + x_p(t)
    \end{equation*}

    Substituting into the non-homogeneous equation gives:
    \begin{align*}
    m \ddot x + b \dot x + k x &= m (\ddot x_h + \ddot x_p) + b (\dot x_h + \dot x_p) + k (x_h + x_p) \\
    &= (m \ddot x_h + b \dot x_h + k x_h) + (m \ddot x_p + b \dot x_p + k x_p) \\
    &= 0 + F_{\text{ext}} = F_{\text{ext}}
    \end{align*}

    Thus, $x(t) = x_h(t) + x_p(t)$ is indeed a solution to the non-homogeneous equation. Since $x_h(t)$ represents the general solution to the homogeneous equation, this form captures all possible solutions to the non-homogeneous equation.
}

\begin{mdframed}
\subsection{Steps to solve a non-homogeneous linear ODE}
\begin{enumerate}
    \item Find the homogeneous solution $x_h(t)$ by solving the characteristic equation.
    \item Find a particular solution $x_p(t)$ using an appropriate method (e.g., undetermined coefficients, variation of parameters).
    \item Combine the homogeneous and particular solutions to get the general solution:
    \begin{equation*}
    x(t) = x_h(t) + x_p(t)
    \end{equation*}
    \item (Optional) Apply initial conditions to find the specific solution if needed.
\end{enumerate}
\end{mdframed}

\subsection{Method of undetermined coefficients}

Suppose that we have a non-homogeneous ODE of the form:
\begin{equation*}
m \ddot x + b \dot x + k x = F_{\text{ext}}(t)
\end{equation*}

We suppose that $F_{\text{ext}}(t) = k y(t)$ where $y(t)$ is a known function. We also consider the case with no damping first ($b=0$):
\begin{equation*}
m \ddot x + k x = k y(t)
\end{equation*}

Rearranging gives:
\begin{equation*}
m \ddot x + k (x - y) = 0
\end{equation*}

We think of $x-y$ as the deviation from the equilibrium position $y$. From a physical perspective, the system will oscillate around the equilibrium position $y$, which sets some natural frequency of oscillation. 
\begin{equation*}
y(t) = A \cos(\omega t)
\end{equation*}

Then inputting wih $\omega_m = \sqrt{\frac{k}{m}}$ gives:
\begin{equation*}
\ddot x + \omega_m^2 x = \omega_m^2 A \cos(\omega t)
\end{equation*}

Seeing the $\cos$ term, we guess a particular solution of the form:
\begin{equation*}
x_p(t) = B \cos (\omega t)
\end{equation*}

Plugging this into the ODE gives:
\begin{align*}
x_p(t) &= B \cos (\omega t) \\
\ddot x_p &= -B \omega^2 \cos (\omega t) \\
-B \omega^2 \cos (\omega t) + \omega_m^2 B \cos (\omega t) &= \omega_m^2 A \cos(\omega t) \\
\end{align*}

This is true if:
\begin{equation*}
\boxed{
B = \frac{\omega_m^2}{\omega_m^2 - \omega^2} A
}
\end{equation*}

where $\omega_m$ is the natural frequency of the system and $\omega$ is the frequency of the external force. We can see that if $\omega$ is close to $\omega_m$, then $B$ becomes very large, which corresponds to the phenomenon of \textbf{resonance}. In reality, with $\omega_m = \omega$, we would need to modify our guess. We will later see how to deal with resonance.

In general, if $F_{\text{ext}}(t) = A \cos (\omega t) + B \sin (\omega t)$ appears to be a sinusoidal function, we can guess:
\begin{equation*}
x_p(t) = C \cos (\omega t) + D \sin (\omega t)
\end{equation*}

Then, we can plug this into the ODE and solve for $C$ and $D$.

\subsection{Exponential inputs}
Similarly, we can have $F_{\text{ext}}(t) = A e^{r t}$, which gives:
\begin{equation*}
m \ddot x + b \dot x + k x = A e^{r t}
\end{equation*}

Then, we can guess a particular solution of the form:
\begin{equation*}
x_p(t) = B e^{r t}
\end{equation*}

Plugging this into the ODE gives:
\begin{align*}
x_p(t) &= B e^{r t} \\
\dot x_p &= r B e^{r t} \\
\ddot x_p &= r^2 B e^{r t} \\
m r^2 B e^{r t} + b r B e^{r t} + k B e^{r t} &= A e^{r t}
\end{align*}

So we can see that to solve for $B$, we can divide both sides by $e^{r t}$ (which is never zero) to get:
\begin{equation*}
B = \frac{A}{m r^2 + b r + k}
\end{equation*}

\nt{
    In general, when we have an $m^{\text{th}}$ order ODE, we can use a similar approach to find particular solutions for exponential inputs. We have:
    \begin{equation*}
        a_m x^{(m)} + a_{m-1} x^{(m-1)} + \cdots + a_1 \dot x + a_0 x = A e^{r t}
    \end{equation*}
    Then, we apply the function:
    \begin{equation*}
        P(r) = a_m r^m + a_{m-1} r^{m-1} + \cdots + a_1 r + a_0
    \end{equation*}
}

We have thus obtained the \textbf{exponential response formula (ERF)}:
\begin{equation*}
\boxed{
    x_p(t) = \frac{A e^{r t}}{p(r)} \quad \text{where } p(r) = a_m r^m + a_{m-1} r^{m-1} + \cdots + a_1 r + a_0 \quad p(r) \neq 0
}
\end{equation*}

\ex{}{
    Find the particular solution of:
    \begin{equation*}
        \ddot x + \dot x + 2x = 4e^{3t}
    \end{equation*}

    Then find the general solution.
}
\sol{}{
    We can identify the homogeneous part of the ODE as:
    \begin{equation*}
        \ddot x + \dot x + 2x = 0
    \end{equation*}
    The characteristic equation is:
    \begin{equation*}
        r^2 + r + 2 = 0
    \end{equation*}
    Solving for the roots gives:
    \begin{equation*}
        r = \frac{-1 \pm \sqrt{1 - 8}}{2} = -\frac{1}{2} \pm i \frac{\sqrt{7}}{2}
    \end{equation*}
    Thus, the homogeneous solution is:
    \begin{equation*}
        x_h(t) = e^{-\frac{1}{2} t} \left( C_1 \cos\left( \frac{\sqrt{7}}{2} t \right) + C_2 \sin\left( \frac{\sqrt{7}}{2} t \right) \right)
    \end{equation*}

    For the particular solution, we can apply the exponential response formula:
    \begin{equation*}
        p(r) = r^2 + r + 2
    \end{equation*}

    Then, evaluating at $r = 3$ gives:
    \begin{equation*}
        p(3) = 3^2 + 3 + 2 = 14
    \end{equation*}

    Thus, the particular solution is:
    \begin{equation*}
        x_p(t) = \frac{4 e^{3t}}{14} = \frac{2}{7} e^{3t}
    \end{equation*}
    
    Finally, the general solution is:
    \begin{equation*}
        \boxed{
            x(t) = e^{-\frac{1}{2} t} \left( C_1 \cos\left( \frac{\sqrt{7}}{2} t \right) + C_2 \sin\left( \frac{\sqrt{7}}{2} t \right) \right) + \frac{2}{7} e^{3t}
        }
    \end{equation*}
}


\ex{}{
    Find the particular solution of:
    \begin{equation*}
        \ddot x + \dot x + 2x = \cos t
    \end{equation*}

}
\sol{}{
    We first complexify the equation as:
    \begin{equation*}
        \ddot z + \dot z + 2z = e^{i t}
    \end{equation*}

    It is possible to do this because the real and imaginary parts of the solution will also be solutions to the original equation, so we can extract the real part at the end to get the particular solution.

    \begin{equation*}
        p(s) = s^2 + s + 2
    \end{equation*}

    Then evaluating at $s = i$ gives:
    \begin{equation*}
        p(i) = i^2 + i + 2 = 1 + i
    \end{equation*}

    Applying ERF gives:
    \begin{equation*}
        z_p(t) = \frac{e^{i t}}{1 + i} = \frac{1 - i}{2} e^{i t} = \frac{1}{2} (1-i) e^{it}
    \end{equation*}

    Thus the particular solution to this complexified equation is:
    \begin{align*}
        z_p(t) &= \frac{1}{2} e^{i t} - \frac{i}{2} e^{i t} \\
        &= \frac{1}{2} (\cos t + i \sin t) - \frac{i}{2} (\cos t + i \sin t) \\
        &= \frac{1}{2} \cos t + \frac{i}{2} \sin t - \frac{i}{2} \cos t + \frac{1}{2} \sin t \\
        &= \frac{1}{2} (\cos t + \sin t) + i\left( \frac{1}{2} (\sin t - \cos t)  \right)
    \end{align*}

    We can then extract the real part to get the particular solution to the original equation:
    \begin{align*}
        x_p(t) &= \text{Re}(z_p(t)) \\
        &= \frac{1}{2} \cos t + \frac{1}{2} \sin t  \\
        &= \frac{\sqrt{2}}{2} \cos\left( t - \frac{\pi}{4} \right)
    \end{align*}

}

\section{Resonance}

An operator takes one function as input and produces another function as output. For example, the operator $D$ takes a function $f(t)$ and produces its derivative $\dot f(t)$. We can also have higher-order operators like $D^2$ which takes a function and produces its second derivative.

The \textbf{identity operator} $I$ is:
\begin{equation*}
I x = x
\end{equation*}

We can take linear combinations of operators. For example:
\begin{align*}
(D^2 + 2D + 2I)x &= \ddot x + 2 \dot x + 2x
\end{align*}

The characteristic polynomial associated with this operator is:
\begin{equation*}
p(s) = s^2 + 2s + 2
\end{equation*}

Meanwhile, $P(D)$ represents the linear combination of operators:
\begin{equation*}
p(D) = D^2 + 2D + 2I
\end{equation*}

Then, we can write the ODE as:
\begin{equation*}
p(D) x
\end{equation*}

Now consider an input function $q(t)$, which gives us the non-homogeneous ODE:
\begin{equation*}
a_n x^{(n)} + a_{n-1} x^{(n-1)} + \cdots + a_1 \dot x + a_0 x = q(t)
\end{equation*}

We extract the characteristic polynomial:
\begin{equation*}
p(s) = a_n s^n + a_{n-1} s^{n-1} + \cdots + a_1 s + a_0
\end{equation*}

Then, we can write the ODE as:

\begin{equation*}
p(D) x = q(t)
\end{equation*}

So the operator $p(D)$ represents the system with $x(t)$ as the response, and the input is represented by $q(t)$. 

Let $Lx = q$ for some operator $L$. Our goal is to find the inverse operator $L^{-1}$ such that:
\begin{equation*}
x = L^{-1} q
\end{equation*}

Recall the ERF which solves $p(D) x = A e^{r t}$ for two conditions:
\begin{enumerate}
    \item $p(r) \neq 0$.
    \item $A$ constant.
\end{enumerate}

We note that if on the LHS we found the characteristic polynomial $p(s)$ with roots $r_1, r_2, \ldots, r_n$, but on the RHS we have $q(t) = A e^{r t}$ where $r$ is one of the roots of $p(s)$, then we have a problem because $p(r) = 0$, which means that the ERF does not apply. This corresponds to the phenomenon of \textbf{resonance}.

So we have that $p(D)x = A e^{rt}$ where $p(r) = 0$. We'd like to make the function $p(D)$ more explicit to find some way to solve for a particular solution.

We repeatedly apply $D$ to $e^{rt}$:
\begin{align*}
I e^{rt} &= e^{rt} \\
D e^{rt} &= r e^{rt} \\
D^2 e^{rt} &= r^2 e^{rt} \\
\vdots \\
D^n e^{rt} &= r^n e^{rt} \\
p(D) e^{rt} &= p(r) e^{rt} = 0
\end{align*}

By the ERF, $x p(t) = \frac{1}{p(r)} e^{rt}$ is a particular solution. But $P(D) e^{rt} = p(r) e^{rt}$ is true even if $p(r) = 0$.

We start from the key identity:
\begin{equation*}
p(D) e^{rt} = p(r) e^{rt}
\end{equation*}

Then we differentiate both sides with respect to $r$:
\begin{equation*}
\frac{\partial}{\partial r} \left[ p(D) e^{rt} \right] = \frac{\partial}{\partial r} \left[ p(r) e^{rt} \right]
\end{equation*}

The operator $p(D)$ does not depend on $r$, so we can pull it out of the derivative:
\begin{equation*}
p(D) \frac{\partial}{\partial r} e^{rt} = p'(r) e^{rt} + p(r) t e^{rt}
\end{equation*}

Since $p(r) = 0$, this simplifies to:
\begin{equation*}
p(D) \frac{\partial}{\partial r} e^{rt} = p'(r) e^{rt}
\end{equation*}

Now we get the partial derivative of $e^{rt}$ with respect to $r$:
\begin{equation*}
\frac{\partial}{\partial r} e^{rt} = t e^{rt}
\end{equation*}

Thus, we have:
\begin{equation*}
p(D) (t e^{rt}) = p'(r) e^{rt}
\end{equation*}

Comparing this to the original equation $p(D) x = A e^{rt}$, we can see that if we let $x = \frac{A}{p'(r)} t e^{rt}$, then we have:
\begin{equation*}
p(D) \left( \frac{A}{p'(r)} t e^{rt} \right) = A e^{rt}
\end{equation*}

Thus, we now have a formula for the particular solution in the case of resonance:
\begin{equation*}
\boxed{
    x_p(t) = \frac{A}{p'(r)} t e^{rt} \quad \text{where } p(r) = 0
}
\end{equation*}

\subsection{Generalization to higher-order ODEs}

We might have a case where $p(r) = 0$, but also $p'(r) = 0$. In this case, we can differentiate the key identity again:
\begin{equation*}
\frac{\partial^2}{\partial r^2} \left[ p(D) e^{rt} \right] = \frac{\partial^2}{\partial r^2} \left[ p(r) e^{rt} \right]
\end{equation*}

This gives us:
\begin{equation*}
p(D) \frac{\partial^2}{\partial r^2} e^{rt} = p''(r) e^{rt} + 2 p'(r) t e^{rt} + p(r) t^2 e^{rt}
\end{equation*}

Since $p(r) = 0$ and $p'(r) = 0$, this simplifies to:
\begin{equation*}
p(D) \frac{\partial^2}{\partial r^2} e^{rt} = p''(r) e^{rt}
\end{equation*}

Now we get the second partial derivative of $e^{rt}$ with respect to $r$:
\begin{equation*}
\frac{\partial^2}{\partial r^2} e^{rt} = t^2 e^{rt}
\end{equation*}

Thus, we have:
\begin{equation*}
p(D) (t^2 e^{rt}) = p''(r) e^{rt}
\end{equation*}

In general, the \textbf{generalized exponential response formula} for the case of resonance is:
\begin{equation*}
\boxed{
    x_p(t) = \frac{A}{p^{(m+1)}(r)} t^{m+1} e^{rt} \quad \text{where } p(r) = p'(r) = \cdots = p^{(m)}(r) = 0
}
\end{equation*}

\ex{}{
    Find the particular solution of:
    \begin{equation*}
    \ddot x - 4x = e^{-2t}
    \end{equation*}
}

\sol{}{
    We first find the homogeneous solution by solving the characteristic equation:
    \begin{equation*}
r^2 - 4 = 0
    \end{equation*}
    which gives us the roots $r = 2$ and $r = -2$. Thus, the homogeneous solution is:
    \begin{equation*}
x_h(t) = C_1 e^{2t} + C_2 e^{-2t}
    \end{equation*}

    We see that $-2$ is a root of the characteristic polynomial, so we have resonance. We can apply the generalized exponential response formula to find the particular solution:
    \begin{equation*}
p(r) = r^2 - 4
    \end{equation*}

    Then we compute $p'(-2)$:
\begin{equation*}
p'(r) = 2r \quad \Rightarrow \quad p'(-2) = -4
\end{equation*}

Thus, the particular solution is:
\begin{equation*}
x_p(t) = \frac{1}{p'(-2)} t e^{-2t} = -\frac{1}{4} t e^{-2t}
\end{equation*}


}

\nt{
    From a physical perspective, although we are at resonance, the particular solution does decay as $t$ increases, which is a consequence of the fact that the input function $e^{-2t}$ also decays as $t$ increases. If we had an input function that did not decay, then we would have a particular solution that grows without bound as $t$ increases, which is the classical notion of resonance.
}


\ex{}{
    Find the particular solution of:
    \begin{equation*}
        \ddot x + 4x = \cos(2t)
    \end{equation*}
}

\sol{}{
    We first find the homogeneous solution by solving the characteristic equation:
    \begin{equation*}
r^2 + 4 = 0
    \end{equation*}
    which gives us the roots $r = 2i$ and $r = -2i$. Thus, the homogeneous solution is:
    \begin{equation*}
x_h(t) = C_1 \cos(2t) + C_2 \sin(2t)
    \end{equation*}

    Complexifying the equation gives us:
    \begin{equation*}
        \ddot z + 4z = e^{2it}
    \end{equation*}

    And hence we see that $2i$ is a root of the characteristic polynomial, so we have resonance. We can apply the generalized exponential response formula to find the particular solution:
    \begin{equation*}
p(r) = r^2 + 4
    \end{equation*}

    Then we compute $p'(2i)$:
\begin{equation*}
p'(r) = 2r \quad \Rightarrow \quad p'(2i) = 4i
\end{equation*}

Thus, the particular solution to the complexified equation is:
\begin{align*}
z_p(t) &= \frac{te^{2it}}{4i}  \\
&= \frac{t}{4i} (\cos(2t) + i \sin(2t)) \\
&= \frac{t}{4} \sin(2t) - \frac{t}{4} i \cos(2t)
\end{align*}

Then we can extract the real part to get the particular solution to the original equation:
\begin{equation*}
x_p(t) = \frac{t}{4} \sin(2t)
\end{equation*}

This solution grows without bound as $t$ increases, which is the classical notion of resonance.

    
}


TBA: reduction of order, variation of parameters, intro to Fourier series